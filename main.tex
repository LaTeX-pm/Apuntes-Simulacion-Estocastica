% Tiene dos configuraciones extras: "twoside", para diferenciar las páginas pares de las impares, y "marginnote", para poder activar las notas al margen.
\documentclass[marginnote,twoside]{notebookFMG}
% \documentclass[]{book}

\usepackage{lipsum}
% Paquete de creación propia. Carga los entornos de los teoremas. Tiene la opción de mostrar colores en el nombre del teorema, si es que se le entrega la opción de "color".
\usepackage[color]{pkg/theoremenvFMG}
%Options: Sonny, Lenny, Glenn, Conny, Rejne, Bjarne, Bjornstrup
\usepackage[Lenny]{fncychap}

%% Datos de la portada
\def\tituloportada {Nuevo Titulo de la Portada}
\def\autordeldocumento {Francisco Muñoz Guajardo}
\def\repositorio {github.com/LaTeX-pm/Template-Apuntes}
\def\fecha {\today}
\def\nombredelcurso {Curso ultra bkn}
\def\codigodelcurso {CO-1234}
\def\nombreuniversidad {Universidad de Chile}
\def\nombrefacultad {Facultad de Ciencias Físicas y Matemáticas}
\def\departamentouniversidad {Departamento de la Universidad}

% Si se desea cambiar la tabla de la portada
\renewcommand{\tablaautor}{
    \begin{tabular}{ll}
        Autor: & \autordeldocumento \\
        Repositorio: & \repositorio \\
        Fecha: & \fecha \\
    \end{tabular}
}

\begin{document}
% Creación de la portada
\portada

% Creación de la tabla de contenidos
\indice

% Configuraciones para el contenido
\contenidoconfig

\chapter{Capítulo de prueba}
\lipsum[1-3]

\section{Sección de prueba}

En esta sección se probarán los entornos de los teoremas
\subsection{Entornos con el estilo \textit{plain}}

\begin{theorem}[Teorema choro]
hola soy un teorema.
\end{theorem}

\begin{lemma}[Lema Choro]
HOLA soy un lema.
\end{lemma}

\begin{corollary}[HOLA soy un corolario]
soy un corolario.
\end{corollary}

\begin{proposition}[Proposición chora]
HoLa Soy una proposición.
\end{proposition}


\subsection{Entornos con el estilo \textit{definition}}

\begin{definition}[EDO no lineal]
una \textit{EDO no lineal} es simplemente una EDO que no es lineal.
\end{definition}

\begin{exercise}[Ejercicio]
Lo anterior es tan trivial que ni siquiera vale como ejercicio.
\end{exercise}

\begin{example}[Inserte un ejemplo aquí]
Hola soy un ejemplo.
\end{example}



\subsection{Entornos con el estilo \textit{remark}}

\begin{remark}[Observación]
Estoy observando el teclado.
\end{remark}

\begin{note}[Nota importante]
Comprar más leche de soja.
\end{note}
\begin{conclusion}[Conclusión final]
En conclusión, me gusta el helado.
\end{conclusion}

\lipsum[4-7]
\subsection{Subsección de prueba}
\lipsum[8]
\subsection{Otra subsección de prueba}
\lipsum[9-20]
\chapter{Otro capítulo de prueba}
\lipsum[10]
\section{Hola soy otra sección de prueba}
\lipsum[11-40]
\end{document}